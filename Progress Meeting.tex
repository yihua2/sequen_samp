% Copyright 2004 by Till Tantau <tantau@users.sourceforge.net>.
%
% In principle, this file can be redistributed and/or modified under
% the terms of the GNU Public License, version 2.
%
% However, this file is supposed to be a template to be modified
% for your own needs. For this reason, if you use this file as a
% template and not specifically distribute it as part of a another
% package/program, I grant the extra permission to freely copy and
% modify this file as you see fit and even to delete this copyright
% notice. 

\documentclass{beamer}
\usepackage{float}
\usepackage{wrapfig}
\usepackage[caption = false]{subfig}
\usepackage{lipsum}
\usepackage{amsfonts}
\usepackage{algorithm}
\usepackage[noend]{algpseudocode}

\usepackage{graphicx}% http://ctan.org/pkg/graphicx
\usepackage{amsmath}% http://ctan.org/pkg/amsmath
\makeatletter
\newcommand{\distas}[1]{\mathbin{\overset{#1}{\kern\z@\sim}}}%
\newsavebox{\mybox}\newsavebox{\mysim}
\newcommand{\distras}[1]{%
  \savebox{\mybox}{\hbox{\kern3pt$\scriptstyle#1$\kern3pt}}%
  \savebox{\mysim}{\hbox{$\sim$}}%
  \mathbin{\overset{#1}{\kern\z@\resizebox{\wd\mybox}{\ht\mysim}{$\sim$}}}%
}
\makeatother
%\usepackage[demo]{graphicx}
% There are many different themes available for Beamer. A comprehensive
% list with examples is given here:
% http://deic.uab.es/~iblanes/beamer_gallery/index_by_theme.html
% You can uncomment the themes below if you would like to use a different
% one:

%\usetheme{default}
\usetheme{CambridgeUS}
\newcommand\Fontvi{\fontsize{18}{7.2}\selectfont}
\newcommand{\lenitem}[2][.7\linewidth]{\parbox[t]{#1}{\strut #2\strut}}
\title{Research Progress  }


\author{Yi Hua  }


% - Use the \inst command only if there are several affiliations.

\date{04/2018}

%\subject{Theoretical Computer Science}
% This is only inserted into the PDF information catalog. Can be left
% out. 

% If you have a file called "university-logo-filename.xxx", where xxx
% is a graphic format that can be processed by latex or pdflatex,
% resp., then you can add a logo as follows:

% \pgfdeclareimage[height=0.5cm]{university-logo}{university-logo-filename}
% \logo{\pgfuseimage{university-logo}

% Delete this, if you do not want the table of contents to pop up at
% the beginning of each subsection:

% Let's get started
\begin{document}
\begin{frame}
\titlepage
\end{frame}


\begin{frame}{Outline}
  \tableofcontents
  % You might wish to add the option [pausesections]
\end{frame}

\section{Bioassay Project}
\subsection{Methodologies}


\begin{frame}{Background of the problem}
\begin{itemize}
    \item In pharmaceutical industries, assays of certain drug substances are taken to ensure quality or to compare the effect of different substances. 
    \item When test samples are plated, sometimes a standard preparation is required in each plate along with all test preparations.
    \item Also, the locations of the test preparations in each plate are of concern -- we hope to assign test preparations to each location as evenly as possible across plates.
   \end{itemize}
    
\end{frame}

\begin{frame}{Problem Setup}
    \begin{itemize}
        \item Example setup: there are 20 plates, 6 test samples and 1 standard preparation, four locations within each plate.
        \begin{figure}
            \centering
            \includegraphics[scale = 0.25]{well_it.png}
            \label{fig:my_label}
        \end{figure}
             \item The model of this problem can be simplified as follows
       \begin{align}
    Y \sim P_{k} + X_{d(k,h)}+ \rho_{h}+\epsilon  
\end{align}
    where 
    Y is the response, X is for test samples or standard preparation (d=0,1..6), $\rho$ is the location effect (h=1..4),and $P$ is plate (k = 1..20).

    \end{itemize}
\end{frame}

\begin{frame}{Plate layout for test/control samples}
    \begin{figure}
        \centering
        \includegraphics[scale=0.5]{design_it.png}
        \label{fig:my_label}
    \end{figure}
\end{frame}




\subsection{Simulation studies}


\section{Multi-armed Bandit Project}
\subsection{Methodologies}
\subsection{Simulation studies}







\begin{frame}{References}
[1]A. Hedayat and Min Yang(2005). Optimal and Efficient Crossover Designs For Comparing Test Treatments with A Control Treatment.\textit{The Annals of Statistics}.Vol. 33. No.2.915-943\\

[2]A. Hedayat(1980). Study of optimality criteria in design of experiments. \textit{International Symposium on Statistics and Related Topics}\\

[3]D. Majumbdar and W.I.Notz(1983), Optimal incomplete block designs for comparing treatments with a control. \textit{The Annals of Statistics}.Vol. 11. No.1.258-266.
\end{frame}
\end{document}